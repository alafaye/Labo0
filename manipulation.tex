\subsection{Manipulation 1: Multimètre}
\paragraph{But}
\paragraph{Méthode}
\paragraph{Résultats}
\subsection{Manipulation 2: Statistique}
\paragraph{But}
Le but de cette manipulation est pouvoir utiliser des outils statistiques afin de représenter graphiquement la précision de la mesure d'une longueur de salle.
\paragraph{Méthode}
La salle a été mesurée 22 fois à l'aide d'une chevilière fixée à l'aide de scotch de carrossier sur une plinthe.
Afin d'augmenter la dispersion des résultats, la mesure a été effectuée en deux fois depuis chaque côté de la salle.
Les 7 premières mesure ont été faites en chosissant un repère au sol (marque de meuble, changement de type de surface).
Pour les 15 suivantes nous avons utilisé des bouts de skotch collés sur le sol et ensuite marqués. Il était ainsi possible de faire les mesures beaucoup plus rapidement en posant les 15 scotchs d'un coup pour créer plus de repères au sol qui pouvaient être indexés.
\paragraph{Résultats}
\subsection{Manipulation 3: Ajustement d'une droite}
\paragraph{But}
L'on va ici tenter d'utiliser des méthodes de régression linéraire afin d'estimer, d'abord de manière approximative et ensuite par calcul les caractéristiques d'une droite passant par une série de points prédéfinis.
\paragraph{Méthode}
Pour la partie approximative se référer à la feuille cadrillée milimetrée en annexe.
Elle comporte les points dessinés et une grossière estimation de la pente et de l'ordonnée à l'origine.
Pour la partie calculatoire, elle a été effectuée avec Excel® qui a donné les valeurs que nous avons obtenues.
\paragraph{Résultats}
\subsection{Manipulation 4: Mesure de $\pi$}
\paragraph{But}
\paragraph{Méthode}
\paragraph{Résultats}
